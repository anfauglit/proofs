\documentclass[letterpaper, 10pt]{article}
\usepackage{amsmath, amsthm, amssymb}
\usepackage{mathtools}
\usepackage{enumitem}

\newtheorem{thm}{Theorem}[section]
\numberwithin{thm}{subsection}
\newtheorem{lem}[thm]{Lemma}
\newtheorem{prop}[thm]{Proposition}
\newtheorem{col}[thm]{Corollary}

\theoremstyle{definition}
\newtheorem{mydef}[thm]{Definition}

\setlength\parindent{0pt}

% Def new command for "less than or equal" relation
\newcommand{\ineq}{%
	\mathrel{\mkern1mu\underline{\mkern-1mu\in\mkern-1mu}\mkern1mu}}

\begin{document}
\setcounter{section}{3}
\section{Natural numbers}

\setcounter{subsection}{3}
\subsection{Order on $\omega$}

\begin{mydef}
	We say that  m is \textbf{less than} n if and only if $m \in n$, whenever m and n are in $\omega$.
\end{mydef}

\begin{prop}
	Let n be a natural number. Then
	\begin{enumerate}[label=(\arabic*)]
		\item if $k \in n$, then $k \subseteq n$;
		\item if $a \in k$ and $k \in n$, then $a \in n$;
		\item if $k \in \omega$, then $k \subseteq \omega$;
		\item if $a \in k$ and $k \in \omega$, then $a \in \omega$.
	\end{enumerate}
\end{prop}

\begin{lem}
	Let $n \in \omega$. Then $n \notin n$.
\end{lem}

\begin{mydef}
	We write $m \ineq n$ if and only if $m \in n$ or $m = n$, whenever m and n are in $\omega$.
\end{mydef}

\begin{lem}
	Let $m \in \omega$ and $n \in \omega$. If $m \ineq n$ and $n \ineq m$, then $m = n$.
\end{lem}

\begin{lem}
	For all $m \in \omega$, we have that $0 \ineq m$.
\end{lem}

\begin{lem}
	For all natural numbers m and n, if $m \in n$, then $m^+ \in n^+$.
\end{lem}

\begin{col}
	For all natural numbers m and n, $m \in n$ if and only if $m^+ \in n^+$.
\end{col}

\begin{thm}[Thrichotomy Law on $\omega$]
	For all m and in in $\omega$, exactly one of the three relationships $m \in n$, $m=n$, $n\in m$ holds.
\end{thm}

\begin{col}
	If m and n are in $\omega$, then $m \in n$ if and only if $m \subset n$.
\end{col}

\begin{thm}[Properties of Inequality]
	For all natural numbers m, n, and p, the following hold:
	\begin{enumerate}[label=(\arabic*)]
		\item $m \in n$ in and only if $m+p \in n+p$.
		\item If $p \ne 0$, then $m \in n$ if and only if $m\cdot p \in n \cdot p$.
	\end{enumerate}
\end{thm}
	
\begin{col}[Cancellation Laws]
	For all natural numbers m, n, and p, the following hold:
	\begin{enumerate}[label=(\arabic*)]
		\item If $m+p = n+p$, then $m=n$.
		\item If $p \ne 0$, then $m \cdot p = n \cdot p$ implies $m=n$.
	\end{enumerate}
\end{col}

\begin{thm}[Well-Ordering Principle]
	Let A be a nonempty subset of $\omega$. Then A has a least element; that is, there is an $l \in A$ such 
	that $l \in a$ for all $a \in A$.
\end{thm}

\begin{thm}[Strong Induction Principle on $\omega$]
	Let A be a subset of $\omega$ and suppose that
	\begin{equation}
		(\forall n \in \omega)(n \subseteq A \to n \in A)
	\end{equation}
	Then $A=\omega$.
\end{thm}

\end{document}
