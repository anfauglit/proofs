\documentclass{article}
\usepackage{amsmath, amsthm, amssymb}
\usepackage{mathtools}
\usepackage{enumitem}

\newtheorem{thm}{Theorem}
\numberwithin{thm}{subsection}
\newtheorem{lem}[thm]{Lemma}
\newtheorem{prop}[thm]{Proposition}

\theoremstyle{definition}
\newtheorem{mydef}[thm]{Definition}

\begin{document}
\setcounter{section}{3}
\section{Natural numbers}

\setcounter{subsection}{3}
\subsection{Order on $\omega$}

\begin{mydef}
	We say that  m is \textbf{less than} n if and only if $m \in n$, whenever m and n are in $\omega$.
\end{mydef}

\begin{prop}
	Len n be a natural number. Then
	\begin{enumerate}[label=(\arabic*)]
		\item if $k \in n$, then $k \subseteq n$;
		\item if $a \in k$ and $k \in n$, then $a \in n$;
		\item if $k \in \omega$, then $k \subseteq \omega$;
		\item if $a \in k$ and $k \in \omega$, then $a \in \omega$.
	\end{enumerate}
\end{prop}

\begin{lem}
	Let $n \in \omega$. Then $n \notin n$.
\end{lem}

\begin{mydef}
	We write $m \leq n$ if and only if $m \in n$ or $m = n$, whenever m and n are in $\omega$.
\end{mydef}

\begin{lem}
	Let $m \in \omega$ and $n \in \omega$. If $m \leq n$ and $n \leq m$, then $m = n$.
\end{lem}

\begin{lem}
	For all $m \in \omega$, we have that $0 \leq m$.
\end{lem}

\begin{lem}
	For all natural numbers m and n, if $m \in n$, then $m^+ \in n^+$.
\end{lem}

\end{document}
